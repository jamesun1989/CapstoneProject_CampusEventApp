\documentclass[onecolumn, draftclsnofoot,10pt, compsoc]{IEEEtran}
\usepackage{url}
\usepackage{setspace}
\usepackage{geometry}

\geometry{textheight=9.5in, textwidth=7in}

% 1. Fill in these details
\def \CapstoneTeamName{		Evently}
\def \CapstoneTeamNumber{		48}
\def \GroupMemberOne{			Sung Kim}
\def \GroupMemberTwo{			Xiaoli Sun}
\def \GroupMemberThree{			Zijian Huang}
\def \CapstoneProjectName{		Campus Events Mobile Application}
\def \CapstoneSponsorPerson{		David Vasquez}

% 2. Uncomment the appropriate line below so that the document type works
\def \DocType{		Technology Review
				%Requirements Document
				%Technology Review
				%Design Document
				%Progress Report
				}
			
\newcommand{\NameSigPair}[1]{\par
\makebox[2.75in][r]{#1} \hfil 	\makebox[3.25in]{\makebox[2.25in]{\hrulefill} \hfill		\makebox[.75in]{\hrulefill}}
\par\vspace{-12pt} \textit{\tiny\noindent
\makebox[2.75in]{} \hfil		\makebox[3.25in]{\makebox[2.25in][r]{Signature} \hfill	\makebox[.75in][r]{Date}}}}
% 3. If the document is not to be signed, uncomment the RENEWcommand below
\renewcommand{\NameSigPair}[1]{#1}

%%%%%%%%%%%%%%%%%%%%%%%%%%%%%%%%%%%%%%%
\begin{document}
\begin{titlepage}
    \pagenumbering{gobble}
    \begin{singlespace}
        \hfill 
        \par\vspace{.2in}
        \centering
        \scshape{
            \huge CS 461 Capstone \DocType \par
            \large Fall 2017
            \vspace{.5in}
            {\large\today}\par
            \vspace{.5in}
            \textbf{\Huge\CapstoneProjectName}\par
            \vfill
            {\large Prepared for}\par
            \vspace{5pt}
            {\Large\NameSigPair{\CapstoneSponsorPerson}\par}
            \vspace{5pt}
            {\large Prepared by }\par
            \vspace{5pt}
            {\Large\NameSigPair{\GroupMemberOne}\par}
            {\large Group  }
            {\Large\NameSigPair{\CapstoneTeamNumber}\par}
            \vspace{5pt}

            % 5. comment out the line below this one if you do not wish to name your team

            \vspace{20pt}
        }
        \begin{abstract}
        This document will be separated into three pieces discussing what technologies will be used for the implementation of the application. The first piece will cover how the application will hold and manipulate data. The second will discuss which tools will be used when designing and prototyping the User Interface. The last piece we will review what framework the application will be developed in.
        \end{abstract}     
    \end{singlespace}
\end{titlepage}
\newpage
\pagenumbering{arabic}
\tableofcontents
\clearpage

\section{Role in project}

Throughout this project, my role will change depending on what needs to be done. A majority of the time I will play as the database administrator with our client David Vasquez to properly manage and manipulate the database. As a database administrator, I will help create the database architecture that is easy to follow and maintain. Depending on how fast the application is completed, the team will move on to developing an administrator website. Everyone's role after this would be switched to a web developer.

% Piece One
\section{Database}
\subsection{Overview}

For any application to work, it needs a location to store large amounts of data. In the case of the Evently Application, we will be storing event data, user information and administrator information. When researching for different databases to use, we ran into the issue of either using a relational database or a NoSQL database. We found two possible relational databases, MySQL and the Oracle Database, and one NoSQL database called MongoDB.

\subsection{Criteria}

The mobile application will be grabbing information from a database and presenting it to a user. Also included will be a search feature that will go through a list of events or groups. So we needed a database that can be easily searched through and has a low volatility. We will also need to be able to group different types of information. Another factor that will be looked at, for all technologies, will be the price.

\subsection{Potential Choices}
\subsubsection{MySQL}

MySQL is an open-source database management system created back in 1995. Many companies still use this tool today, such as Uber and Pinterest. You can set up various rules for the database that will help manage and manipulate each table \cite{MySQL}. Another reason to use this tool is a powerful feature called joining. Joining combines information from two or more tables into one, which we will need for this application. This tool was taken into consideration because our client has already set up a small MySQL database for the application.

\subsubsection{MongoDB}

MongoDB is an open-source NoSQL database. NoSQL databases are great for storing large amounts of data without having to worry about any type of structure. They store data fields and values together into one record, which makes data retrieval faster \cite{MySQLvsMongo}.

\subsubsection{Amazon Aurora}

The Amazon Aurora is a relational database engine designed for the cloud. This tool is compatible with both MySQL and PostgreSQL databases.The speed of the tool is what really stands out and according to the website, "Aurora is up to five times faster than standard MySQL databases and three times faster than standard PostgreSQL database \cite{Amazon}."

\subsection{Discussion}

When making a decision between these three products, it came down to a decision between a relational and NoSQL database. Relational databases are easy to maintain with a wide variety of useful features to help manage your database. NoSQL database can handle high transaction loads because of it's horizontal scaling \cite{MySQLvsMongo}. The application does not require millions of changes in a short period of time since these changes will be managed by a single admin per group. To add on to the fact, joining tables is a much-needed feature of the project which is a lot simpler in a relational database. As for comparing the relational databases, Amazon Aurora and MySQL, came down to the cost. Aurora is not an open-source tool, meaning we will have to pay to use this service.

\subsection{Conclusion}

Of the three products presented, MySQL seems to be the best choice for this application. NoSQL database is too unorganized for the content the application is providing and Amazon Aurora is not free to use. 

\clearpage
% Piece Two
\section{Wireframe Tools}
\subsection{Overview}

Designing a simple and easy to use interface is quite possibly the biggest portion of creating a new application. There are multiple tools that anyone can use to make a prototype of a mobile application. These tools help accelerate development process by providing a visual that the developers can work towards. Our initial designs were all drawn by hand from our client. I recommended one of these tools so he can show us a clear picture of how the application should look.

\subsection{Criteria}

When using one of these tools, we are looking for something simple and easy to learn. It should have the ability to create mock applications that mimic functionality. Finally. the ability to share the prototype you created, so the rest of the team can see it.

\subsection{Potential Choices}
\subsubsection{Moqups}

Moqups is a web-based wireframing tool that lets you create user interface designs. It gives you the ability to create functional prototypes with a wide variety of features you can add to the application \cite{Moqups}. Moqups also provides templates for both Android and iOS platforms. The first 300 stencils are free, but can later be upgraded if you sign up with an annual plan.

\subsubsection{InVision}

Just like Moqups, InVision provides users a powerful design prototyping tool. It also allows you to create a functional prototype that you can use to show off the application. The collaboration feature really stands out on this tool by creating a direct connection between team members and clients to receive comments directly \cite{invision}. 

\subsubsection{Wireframe.cc}

Wireframe.cc provides a more simple interface for creating mock ups. Works like a sketchbook/paint that provides basic things like scroll bars and text boxes for you to add to your template \cite{wireframe}. It gives you the option to change what type of device you are on, such as a web browser or on your phone.  

\subsection{Discussion}

Compared to Moqups and InVision, wireframe.cc was a little too simple for what needed to be done. The reason Moqups and Invision stood out the most is that of their ability to create functional prototypes. Preparing a live demo for your client provides more value to them, and you will receive a more appropriate response to what needs to be added or edited. When comparing Moqups and InVision, InVision seems like it's the way to go. More reputable companies and have gone and commented on Invision. Companies like Twitter, Uber, and Nike have stated how creating these designs beforehand make them great goals to work towards. 

\subsection{Conclusion}

Picking UI prototyping tools is normally based on preference. The first UI design was drawn on paper, which made it a little hard to see, so I recommended Moqups to my team when we were trying to think of a design. After researching about possible alternatives, InVision really stuck out to me. The example prototypes looked a lot cleaner and they have many positive comments from reputable companies. 

\clearpage
% Piece Three
\section{Development Framework}
\subsection{Overview}

A framework provides a structure to support the development of applications made for a specific environment. In our case, we would need a framework that is able to create mobile applications. Using frameworks speed up the process of development by providing easy access to tools and features. 

\subsection{Criteria}

The key component we are looking for is the ability to develop an application on both Android and iOS platforms. The initial requirement for this project was to develop in either one, but we decided to create it for both. Another feature we would like to have is the application being as close to a native application as possible. This will ensure that the application runs smoothly and without a lot of error. 

\subsection{Potential Choices}
\subsubsection{React Native}
React Native is a JavaScript framework for writing native applications. With this framework, you can create applications for both Android and iOS based on the same codebase. This framework is based on the React JavaScript library \cite{react}. The difference between the two is that React Native is used for building mobile applications. Some companies that use this framework for their application include Facebook, Instagram, and Tesla.
\subsubsection{Iconic}

Iconic is an HTML5 mobile application framework used to write hybrid applications. Hybrid applications are essentially websites on a mobile platform. Iconic can be styled to make the application look like a native application. Development in this framework can also be achieved using CSS or JavaScript regardless of the operating platform \cite{iconic}. 
\subsubsection{PhoneGap}

Another hybrid application development framework, PhoneGap, usually isn't paired by itself. This framework uses the Cordova plugins to bridge together websites and mobile devices \cite{phonegap}. Some examples of native API's include GPS, camera, notifications, etc.

\subsection{Discussion}

The Iconic and PhoneGap frameworks are usually paired together to make the application look and feel like a native app \cite{iconic}. In contrast, React Native will create applications very similar to native applications, which will have access to more of the native features without restriction. Native applications usually perform better than their hybrid counterparts but can be harder to implement.
\subsection{Conclusion}

The React Native framework provides us a bit more of a challenge when developing the application, but the challenge is well worth it since we are creating a better performing app. We decided to stick with the React Native framework to gain more experience with the technology.

\clearpage
\bibliographystyle{IEEEtran}
\bibliography{references}

\end{document}
