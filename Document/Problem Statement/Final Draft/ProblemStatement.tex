\documentclass[10pt,letterpaper]{article}

\usepackage{color}
\usepackage{url}
\usepackage{hyperref}
\usepackage{enumitem}
\usepackage{geometry}
\geometry{left=0.75in, right=0.75in, top=0.75in, bottom=0.75in}

\def\name{Group 48}


\begin{document}
\begin{titlepage}
\begin{center}
    \Huge
    \textbf{Problem Statement}
    
    \vspace{0.5in}
    \large
    CS461 - CS Senior Capstone\\
    
    \vspace{0.2in}
    \large
    Fall 2017\\
    
    \vspace{0.2in}
    \large
    Instructor: D. Kevin McGrath, Kirsten Winters\\
    
    \vspace{0.2in}
    \textbf{Group 48: Xiaoli Sun, Sung Kim, Zijian Huang}
    
    \vspace{0.2in}
    \textbf{Client: David Vasquez}
    
    \vspace{0.5in}
    \textbf{Abstract}\\
    \vspace{0.2in}
    
    Every day, Oregon State University holds many events around campus. Big events like football games to small events like club meetings. Anyone who wants to be involved around campus would hate if they missed out, so David Vasquez proposed a solution for this issue. The campus events mobile application will provide students and community a neat and organized place to find all the events around campus. This application will work with various groups around OSU to retrieve information on the events they will hold. All that data will be stored on a MySQL server. Another feature of this product will include a secure login to provide users a safe browsing experience. The Campus Events application will be available on both Android and IOS devices.
    
    \vspace{0.3in}
    \vfill
    %\Large 
    
    Oct 9, 2017

\end{center}
\end{titlepage}

\newpage

\begin{center}
\large
\textbf{Problem Description}
\end{center}

The most important feature of this application will be a secure login, to keep those with malicious intents from editing user data. On top of that we want to build an application that runs on both Android and IOS. This was not required, but since we want to reach a large audience, we started looking for a solution to the problem. When researching for languages to develop an IOS application we would use either Objective C or Swift. If we were writing just for IOS, Swift would have been preferred since it more closely resembles natural English and seemed like the fastest to learn. As for Android devices, we would have used Java or Kotlin. Java is the primary language used to write an Android application, but Kotlin, a language from JetBrains, received first party support from Google. Another issue is how will we store all the data from the groups that will be using this application. Will we be using a SQL database or a NoSQL database? The final and one of the most important problems would be creating a simple and efficient user interface. David Vasquez, the project owner, brought some ideas but the current user interface he designed is not set in stone. \\

\begin{center}
\large
\textbf{Proposed Solution}
\end{center}

To tackle the security issue for the application, we will be using the Oauth 2.0 framework for a secure login. This framework follows a series of steps before allowing any data to be displayed. Users will first be asked for their user name and password. After their credentials have been verified, the server will send the application an access token to allow the user to see the data. To have the application run on both IOS and Android, David proposed that we write the application using React Native. React Native is an open source cross-platform development framework that allows developers to create a single application from the same code base. React Native also provides their own mobile application so you can easily run your project on your own physical device instead of having to set up a virtual machine. React Native is new to the team, so we are currently researching how to use it. As for storing the data, David and Sung Kim have experience creating databases using MySQL. Finding a solution for how David wants the application to look was a little challenging, since this product will not be a native Oregon State University application. The team will also come up with a logo since it currently does not have one. Sung Kim provided David with a web application that you can use to create mock ups of your desired user interface. His friend is also a graphic designer and will also help with the design. As of 10/13/2017 the application will feature four tabs at the bottom of the screen. The first tab will be an event feed to see upcoming events for groups you’ve followed. The second tab will be the “Discover” feature, which will allow you to find and follow various groups. The last two pages will be the “Manage Following” and the user profile page. \\

\begin{center}
\large
\textbf{Performance Metric}
\end{center}

To properly track the progress of our product, we will divide each portion of the project into milestones:

\begin{enumerate}
   \item Development of a small database
   \begin{itemize}
     \item We will include static data for testing the application.
     \item The database will become dynamic when we receive data from the groups around campus.
   \end{itemize}
   \item A functional user interface
   	\begin{itemize}
    	\item A secure log-in landing page.
        \item A profile page to display user data which can be revised and edited.
        \item A list of groups or organizations that the user is following.
        \item A "Discover" page that will display a list of groups and organizations the user can search and follow.
        \item The "Events feed" page will feature a calendar and list view of events are are coming up.
   \end{itemize}
   
These milestones are subject to change as David will add and remove milestones as he sees fit. This project will be completed when group 48 delivers a functional mobile application on both Android and IOS. 
\end{enumerate}

\end{document}
