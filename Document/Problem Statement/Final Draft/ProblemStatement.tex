\documentclass[10pt,letterpaper]{article}

\usepackage{color}
\usepackage{url}
\usepackage{hyperref}
\usepackage{enumitem}
\usepackage{geometry}
\geometry{left=0.75in, right=0.75in, top=0.75in, bottom=0.75in}

\def\name{Group 48}


\begin{document}
\begin{titlepage}
\begin{center}
    \Huge
    \textbf{Problem Statement}
    
    \vspace{0.5in}
    \large
    CS461 - CS Senior Capstone\\
    
    \vspace{0.2in}
    \large
    Fall 2017\\
    
    \vspace{0.2in}
    \large
    Instructor: D. Kevin McGrath, Kirsten Winters\\
    
    \vspace{0.2in}
    \textbf{Group 48: Xiaoli Sun, Sung Kim, Zijian Huang}
    
    \vspace{0.2in}
    \textbf{Client: David Vasquez}
    
    \vspace{0.5in}
    \textbf{Abstract}\\
    \vspace{0.2in}
    
    Every day, Oregon State University holds many events around campus. Big events like football games to small events like club meetings. Anyone who wants to be involved around campus would hate if they missed out, so David Vasquez proposed a solution for this issue. The campus events mobile application will provide students and community a neat and organized place to find all the events around campus. This application will retrieve data from various groups around the community and display the information on your phone. 
    \vspace{0.3in}
    \vfill
    %\Large 
    
    Oct 9, 2017

\end{center}
\end{titlepage}

\newpage

\begin{center}
\large
\textbf{Problem Description}
\end{center}

The key features of this application will include a secure login, a simple and effective interface, and compatibility for both Android and IOS. A secure login will ensure group administrators that their data will not be altered by individuals with malicious intents. Creating a user-friendly interface will allow anyone to download and immediately start using the application. And to ensure a wide audience will be reached, the team will develop the application on both Android and IOS devices. \\

\begin{center}
\large
\textbf{Proposed Solution}
\end{center}

To create an application with a secure login, we will use an industry standard authentication framework called OAuth 2.0. This framework follows a series of steps before allowing any data to be displayed. Users will first be asked for their user name and password. After their credentials have been verified, the server will send the application an access token to allow the user to see the data. To create a simple and effective user interface, the UI must be modern and easy to understand. Sung Kim has provided David Vasquez an online application that will create mock-ups of any application you want to create. This will be a foundation for how we want to design the product and ultimately how it will look. As for creating an application for both Android and IOS, our client recommended we use a development framework called React Native. This tool allows us to create an application on android and IOS using the same code base. Normally you would have to write two separate applications, since IOS and Android use different languages.\\

\begin{center}
\large
\textbf{Performance Metric}
\end{center}

To properly track the progress of our product, we will divide each portion of the project into milestones. These milestones are subject to change as David will add and remove milestones as he sees fit:

\begin{enumerate}
   \item Secure login page
     \begin{itemize}
     	\item Set as landing page for the application
        \item Interface is a simple username and password verification
     \end{itemize}
   \item User Interface
     \begin{itemize}
        \item A logo has been created
        \item The name for the application has been decided
        \item The "Events Feed" tab displays a list or calender of events that are approaching
        \item The "Discover" tab displays a list of groups you can follow
        \item The "Following" tab let's the user manage their current interests
        \item The "Profile" feature displays and edits your current account settings
	\end{itemize}
   \item IOS and Android compatibility 
   	\begin{itemize}
    	\item Android application is complete
    	\item IOS application is complete
    \end{itemize}

\end{enumerate}

\end{document}