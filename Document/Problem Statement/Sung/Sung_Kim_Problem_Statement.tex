\documentclass[10pt]{article}
\usepackage[letterpaper,portrait,margin=0.75in]{geometry}
\begin{document}
\newcommand\tab[1][1cm]{\hspace*{#1}}
\begin{titlepage}

	\centering
	\scshape
	\vspace*{\baselineskip}
	
	%	Title

	\vspace{0.75\baselineskip} 
	
	{\LARGE Campus Events Mobile Application Problem Statement} 
	
	\vspace{6.25\baselineskip}
	
	{\scshape\Large Sung Kim \\ CS 461 \\ Fall 2017 \\} 
	
	\vspace{5\baselineskip} 
	
	\tab The campus events mobile application was proposed by David Vasquez to help students and member of the community to get easily find events you can attend. The challenge for this project is creating a mobile application that works on both Android and IOS platforms. Doing this will allow almost everyone with a smart phone to become more involved with the community.
	
	\vspace{1\baselineskip}
	
	\tab We face this problem using the development language React Native which creates applications on both platforms. Also part of our project will be working with various groups around the OSU campus and the Corvallis community to gain data on the events that will be placed inside a SQL database.
	\vfill 
	
	\vspace{0.3\baselineskip} 
	
\end{titlepage}

\centerline{Project Problem}
	\vspace{1\baselineskip}
	
	\tab For this project one of the challenges will be to create an application to that runs on both Android and IOS. This was not required, but since we want to reach a large audience, we started looking for a solution to the problem. When researching for languages to develop an IOS application we would use either Objective C or Swift. If we were writing just for IOS, Swift would have been preferred since it more closely resembles natural English and seemed like the fastest to learn. As for Android devices, we would have used Java or Kotlin. Java is the primary language used to write an Android application, but Kotlin, a language from JetBrains, received first party support from Google. Another issue is how will we store all the data from the groups that will be using this application. Will we be using a SQL database or a NoSQL database? The final and one of the most important problems would be how will the application look like. David Vasquez, the project owner, brought some ideas but the current user interface he designed is not set in stone. 

	\vspace{1\baselineskip}
\centerline{Proposed solution}
	\vspace{1\baselineskip}\
	
	\tab A solution for the first issue was to use a development language that works for both an Android and IOS platform. David proposed that we write the application using React Native. React Native also provides their own application so you can easily run your project on your own physical device instead of having to set up a virtual machine. React Native is new to the team, so we are currently researching the language. As for storing the data, David and I have experience creating databases using SQL. Finding a solution for how David wants the application to look was a little challenging. Since this application will not be a native Oregon State University application thinking, of a color scheme is the first task. I provided David with a web application that you can use to create mock ups of your desired user interface. His friend is also a graphic designer and he will also help with the design. 
	
	\vspace{1\baselineskip}
\centerline{Performance metrics}
	\vspace{1\baselineskip}
	
	\tab To check the progress of our project, we will divide each portion of the project into different parts. For the user interface, we will have each feature be a milestone. This will constantly change depending on what David wants to add or remove. When the features of the application are changing, so will the code that’s running in the background. Milestones for the database will include making sure the tables are designed efficiently and each group will have nightly updates for their events. This project will be concluded when group 48 delivers a working application on both Android and IOS.  



\end{document}